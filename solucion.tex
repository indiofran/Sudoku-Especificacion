\documentclass[a4paper]{article} 

\setlength{\parskip}{0.1em}
\input{Algo1Macros}
\usepackage{caratula} % Version modificada para usar las macros de algo1 de ~> https://github.com/bcardiff/dc-tex

\begin{document}

\titulo{TP de Especificaci\'on}
\subtitulo{Sudoku}
\fecha{1 de Abril de 2017}
\materia{Algoritmos y Estructuras de Datos I}
\grupo{Grupo 10}

% Pongan cuantos integrantes quieran
\integrante{Gomez Salaverri, Francisco}{550/15}{francisco@gomezsalaverri.com}
\integrante{Matias Colque, Nadia Noemí}{188/17}{nmatias@dc.uba.ar}
\integrante{Girón, Jorge David}{637/16}{jorgedavid2905@gmail.com}
\integrante{Boncini Oubel, Fernando Gabriel}{823/15}{fboncini@dc.uba.ar}

\maketitle

\section{Problemas}
1.  \begin{proc}{sudoku\_esTableroValido}{\textbf{\In t}: \matriz{\ent}, \textbf{\Out result}: \bool}{}{}
    \pre{\True}
    \post{tableroValido(t) = \textbf{result}}
    \pred{esMatrizNuevePorNueve}{t: \matriz{\ent}}{\\
        length(t)=9 $\land$\\
        (\forall j: \ent)(enRango(t,j) \implicaLuego (length(t[j])=9))\\
        }
    \pred{elementosDelCeroAlNueve}{t: \matriz{\ent}}{\\
        (\forall i: \ent)(\forall j: \ent)((enRango(t, i) $\yLuego$ enRango(t[i], j)) $\implicaLuego$ (0 \leq t[i][j] \leq 9))\\
        }
    \pred{tableroValido}{t: \matriz{\ent}}{\\
        esMatrizNuevePorNueve(t) $\land$\\
        elementosDelCeroAlNueve(t)\\
        }
    \end{proc}


2. \begin{proc}{sudoku\_esCeldaVacia}{\textbf{\In t}: \matriz{\ent}, \textbf{\In f}: \ent, \textbf{\In c}: \ent,     \textbf{ \Out result}: \bool }{}{}
      \pre{tableroValido(t) $\land$ esFilaYColumnaValida(f,c)}
      \post{\textbf{result} = celdaVacia(t,f,c)}
      \pred{celdaVacia}{t: \matriz{\ent}, i: \ent, j: \ent }{t[i][j] = 0}  
      
   \end{proc}

3. \begin{proc}{sudoku\_nroDeCeldasVacias}{\textbf{\In t}: \matriz{\ent}, \textbf{\Out result} : \ent }{}{}
		\pre{tableroValido(t)}
    \post
		{
			\textbf{result} = nroCeldasVacias(t)
		}
			\aux{nroCeldasVacias}{s: \matriz{\ent}}{\ent}
			{\\
				(\forall i: \ent)(\forall j: \ent) enRango(s,i) $\yLuego$ enRango(s[i],j) $\implicaLuego$ \\
				\sum_{i=0}^{length(s)-1}\sum_{j=0}^{length(s[j])-1} \IfThenElse{celdaVacia(s,i,j)}{1}{0}
			}
		
	\end{proc}

4. \begin{proc}{sudoku\_primeraCeldaVaciaFila}{\textbf{\In t}: \matriz{\ent}, \textbf{\Out result} : \ent  }{}{}
		\pre{tableroValido(t)}
    \post
		{
			\IfThenElse{celdasVacias(t) = 0 }
			{-1}
			{
				(\exists i: \ent)(\exists j: \ent) \textbf{result} = i $\land$ enRango(t,i) $\yLuego$ enRango(t[i],j)        $\yLuego$ celdaVacia(t,i,j) $\land$ menorFilaVacia (t,i) $\land$ menorColumnaDeLaFilaVacia(t,i,j)
				
				\pred{menorFilaVacia}{\textbf{t:} \matriz{\ent}, \textbf{i:} \ent}
				{\\
					(\forall f: \ent)(\forall g: \ent) enRango(t,f) $\yLuego$ enRango(t[f],g)\\
					$\implicaLuego$ celdaVacia(t,f,g) $\land$ f \geq i)
				}
		
			\pred{menorColumnaDeLaFilaVacia}{\textbf{t:} \matriz{\ent}, \textbf{i:} \ent, \textbf{j:} \ent}
			{\\
				(\forall g: \ent) enRango(t[i],g)\\
				$\land$ celdaVacia(t,i,g) $\implicaLuego$ g \geq j)
			}
			}
		}
	\end{proc}

5. \begin{proc}{sudoku\_primeraCeldaVaciaColumna}{\textbf{\In t}: \matriz{\ent}, \textbf{\Out result} : \ent }{}{}
		\pre{tableroValido(t)}
    \post
		{
			\IfThenElse{celdasVacias(t) = 0 }
			{result = -1}
			{
				(\exists i: \ent)(\exists j: \ent) \textbf{result} = j $\land$ enRango(t,i) $\yLuego$ enRango(t[i],j)        $\yLuego$ celdaVacia(t,i,j) $\land$ menorFilaVacia (t,i) $\land$ menorColumnaDeLaFilaVacia(t,i,j)
			}
		}
	\end{proc}

6. \begin{proc}{sudoku\_valorEnCelda}{\textbf{\In t}: \matriz{\ent}, \textbf{\In f}: \ent, \textbf{\In c}: \ent, \textbf{\Out result}: \ent}{}{}
		\pre{tableroValido(t) $\land$ \\
					esFilaYColumnaValida(f,c) $\land$ \\
					celdaVacia(t[f][c]) = false \\
		}
    \post{\textbf{result} = t[f][c]}
	\end{proc}

7. \begin{proc}{sudoku\_llenarCelda}{\textbf{inout t}: \matriz{\ent} \textbf{\In f}: \ent, \textbf{\In c}: \ent, \textbf{\In value}: \ent}{}{}
    \pre
		{
			tableroValido(t) $\land$\\
			esFilaYColumnaValida(f,c) $\land$\\
			1 \leq value \leq 9 $\land$\\
			t = t_{0} $\land$\\
			t_{0}[f][c] = 0
		}
    \post{ t = SetAt(t_{0}[f], c, value)}
	\end{proc}

8. \begin{proc}{sudoku\_vaciarCelda}{\textbf{inout t:} \matriz{\ent}, \textbf{\In f}: \ent, \textbf{\In c}: \ent, \textbf{\Out result:} \bool}{}{}
		\pre
		{
			tableroValido(t)  $\land$ \\
			esFilaYColumnaValida(f,c)  $\land$ \\
			(t[f][c] \neq 0)$\land$
			t = t_{0} 
			
		}
    \post
		{\textbf{t} = SetAt(t_{0}[f], c, 0) \\
			 
		}
	\end{proc}

9. \begin{proc}{sudoku\_esTableroParcialmenteResuelto}{\textbf{\In t}: \matriz{\ent}, \textbf{\Out result}: \bool }{}{}
	\pre{\True}
        \post{\textbf{result} = TableroParcialmenteResuelto(t)}
	\pred{noHayRepetidos}{s: \TLista{\ent}}{\\
        (\forall i: \ent)(\forall j: \ent)((enRango(s, i) $\land$ enRango(s, j) $\land$
        j \neq i)) $\implicaLuego$ ((s[i] = 0 $\land$ s[j]=0) $\oLuego$ s[i] \neq s[j])\\
        }
	\pred{TableroConElementosDelCeroalNueve}{t: \matriz{\ent}}{\\
        (\forall i: \ent)(\forall j: \ent)((enRango(t, i) $\yLuego$ enRango(t[i], j))$\yLuego$
        (0 \leq t[i][j] \leq 9) \\
        }
	\pred{FiladeTableroParcialmenteResuelto}{t: \matriz{\ent}}{\\
	TableroConElementosDelCeroalNueve(t) $\land$\\
        (\forall i: \ent)(enRango(t, i) $\implicaLuego$ noHayRepetidos(t[i])\\
        }	
	\pred{ColumnadeTableroParcialmenteResuelto}{t: \matriz{\ent}}{\\
	TableroConElementosDelCeroalNueve(t)$\land$\\
        (\forall i: \ent)(\forall j: \ent)(\forall h: \ent)((enRango(t, i)  $\land$ enRango(t, j) $\yLuego$ enRango(t[i], h)$\land$ 
	i \neq j)  $\implicaLuego$ ((t[i][h]=0 $\lnad$ t[j][h]=0) $\oLuego$ (t[i][h] \neq t[j][h]))\\
        }	
	\pred{regiondeTableroParcialmenteResuelto}{t: \matriz{\ent}}{\\
        (\forall i: \ent)(\forall j: \ent)((enRango(t, i) $\land$ i mod 3 = 0) $\yLuego$
        (enRango(t[i], j) $\land$ j mod 3 = 0)\\
	$\implicaLuego$ (\exists s \TLista{\ent})(s = Concat(Concat(subseq(t[i], j, j+3), subseq(t[i+1], j, j+3)), subseq(t[i+2], j, j+3))$\land$\\
	noHayRepetidos(s))\\
        }
	\pred{TableroParcialmenteResuelto}{t: \matriz{\ent}}{\\
        TableroValido(t) $\land$\\
	filadeTableroParcialmenteResuelto(t) $\land$\\
	columnadeTableroParcialmenteResuelto(t) $\land$\\
	regiondeTableroParcialmenteResuelto(t)\\
        }
   \end{proc}


10. \begin{proc}{sudoku\_esTableroTotalmenteResuelto}{\textbf{\In t}: \matriz{\ent}, \textbf{\Out result}: \bool}{}{}
		\pre{tableroValido(t)}
    		\post{\textbf{result} = tableroTotalmenteResuelto(t)}
		\pred{tableroTotalmenteResuelto}{t: \matriz{\ent}}{\\
	nroCeldasVacias(t) = 0 $\land$\\
	tableroParcialmenteResuelto(t)
	}
	\end{proc}


11. \begin{proc}{sudoku\_esSubTablero}{\textbf{\In t_{0},t_{1}}: \matriz{\ent}, \textbf{\Out result}: \bool}{}{}
		\pre{tableroValido(t_{0}), tableroValido(t_{1})}
    \post
		{
		result = esSub(t_{0},t_{1})
		}
				\pred{esSub}{t_{0}: \matriz{\ent},t_{1}: \matriz{\ent}}{\\
					(\forall i: \ent)(\forall j: \ent) length(t_{0}) = length(t_{1}) $\yLuego$\\
					enRango(t_{0},i) $\yLuego$ length(t_{0}[i]) = length(t_{1}[i]) $\land$ \\
					enRango(t_{0}[i],j) $\implicaLuego$ t_{0}[i][f] \neq 0 $\land$ t_{0}[i][f] = t_{1}[i][f]
        
		}
	\end{proc}

12. \begin{proc}{sudoku\_tieneSolucion}{\textbf{\In t}: \matriz{\ent}, \textbf{\Out tienesolucion}: \bool}{}{}
		\pre{tableroValido(t)}
    \post{
		\textbf{tienesolucion} = ((\exists x: \matriz{\ent}) esSolucion(t,x))\\
		}
		\pred{esSolucion}{t: \matriz{\ent},x: \matriz{\ent}}
		{\\
			esSub(t,x) $\yLuego$ tableroTotalmenteResuelto(x)
		}
		\hspace*{20mm}\aux{solucion}{t: \matriz{\ent},x: \matriz{\ent}}{\matriz{\ent}}
			{\\
				\hspace*{20mm}\IfThenElse{esSolucion(t,x)}
				{x}
				{t}
	\end{proc}\\
	


13. \begin{proc}{sudoku\_resolver}{\textbf{inout t}: \matriz{\ent}, \textbf{\Out tienesolucion}: \bool}{}{}
		\pre{t_{0}=t}
    \post
		{
		\IfThenElse {(esTableroValido(t) $\land$ (\exists x: \matriz{\ent})(t_{0} \neq solucion(t_{0},x) $\land$ esSolucion(t_{0},x))}
		{tieneSolucion = True $\land$ t = solucion(t_{0}, x)}
		{tieneSolucion = False $\land$ t = t_{0}}
		}
	\end{proc}


14. \begin{proc}{sudoku\_copiarTablero}{\textbf{\In src}: \matriz{\ent}, \textbf{\Out target}: \matriz{\ent}}{}{}
       \pre{\True}
       \post
			{
			src = target
			}
    \end{proc}



\section{Predicados y Auxiliares generales}


\aux{Aux}{i: \ent}{\bool}{\True}
		
\pred{esFilaYColumnaValida}{i: \ent, j: \ent }{0 \leq i, j \leq 8}

		\pred{enRango}{t: \TLista{t}, i:\ent}
		{\\
			0 \leq  i < length(t)\\
		}
		
	
		
        
		
\section{Decisiones tomadas}

\end{document}
