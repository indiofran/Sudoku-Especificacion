\documentclass[a4paper]{article} 

\setlength{\parskip}{0.1em}
\input{Algo1Macros}
\usepackage{caratula} % Version modificada para usar las macros de algo1 de ~> https://github.com/bcardiff/dc-tex

\begin{document}

\titulo{TP de Especificaci\'on}
\subtitulo{Sudoku}
\fecha{1 de Abril de 2017}
\materia{Algoritmos y Estructuras de Datos I}
\grupo{Grupo 10}

% Pongan cuantos integrantes quieran
\integrante{Gomez Salaverri, Francisco}{550/15}{francisco@gomezsalaverri.com}
\integrante{Matias Colque, Nadia Noemí}{188/17}{nmatias@dc.uba.ar}
\integrante{Girón, Jorge David}{637/16}{jorgedavid2905@gmail.com}
%\integrante{Apellido, Nombre4}{004/01}{email4@dominio.com}

\maketitle

\section{Problemas}
1. \begin{proc}{sudoku\_esTableroValido}{\textbf{\In t}: \matriz{\ent}, \textbf{\Out result}: \bool}{}{}
    \pre{\True}
    \post{tableroValido(t) = \textbf{result}}
		
				\pred{tableroValido}{t: \matriz{\ent}}
				{\\
        esFilaValida(t)  $\land$  esColumnaValida(t)\\
        }
				
				\pred{esFilaValida}{t: \matriz{\ent}}
				{\\
        (\forall i: \ent) (\forall j: \ent)  enRango(t,i)  $\yLuego$\\
				enRango(t[i],j)  $\yLuego$  length(t[i]) = 9  $\implicaLuego$  0 \leq  t[i][j]  \leq 9\\
				}
				
				\pred{esColumnaValida}{t: \matriz{\ent}}
				{\\
        (\forall i: \ent) (\forall j: \ent) length(t) = 9  $\land$ enRango(t,i) $\yLuego$\\
				enRango(t[i],j)   $\implicaLuego$ 0  \leq  t[i][j] \leq 9\\
				}
	\end{proc}


2. \begin{proc}{sudoku\_esCeldaVacia}{\textbf{\In t}: \matriz{\ent}, \textbf{\In f}: \ent, \textbf{\In c}: \ent, \textbf{ \Out result}: \bool }{}{}
      \pre{tableroValido(t) $\land$\\
      0 \leq f, c \leq }
      \post{\textbf{result} = celdaVacia(f,c)
				\pred{celdaVacia}{t: \matriz{\ent}, i: \ent, j: \ent }{s[i][j] = 0}
			}
	\end{proc}

3. \begin{proc}{sudoku\_nroDeCeldasVacias}{\textbf{\In t}: \matriz{\ent}, \textbf{\Out result} : \ent }{}{}
		\pre{tableroValido(t)}
    \post
		{
			\textbf{result} = nroCeldasVacias(t)
			\aux{nroCeldasVacias}{s: \matriz{\ent}}{\ent}
			{\\
				(\forall i: \ent)(\forall j: \ent) enRango(s,i) $\yLuego$ enRango(s[i],j) $\implicaLuego$ \\
				\sum \IfThenElse{celdaVacia(s,i,j)}{1}{0}
			}
		}
	\end{proc}

4. \begin{proc}{sudoku\_primeraCeldaVaciaFila}{\textbf{\In t}: \matriz{\ent}, \textbf{\Out result} : \ent  }{}{}
		\pre{tableroValido(t)}
    \post
		{
			\IfThenElse{celdasVacias(t) = 0 }
			{-1}
			{
				(\exists i: \ent)(\exists j: \ent) \textbf{result} = i $\land$ enRango(t,i) $\yLuego$ enRango(t[i],j)        $\yLuego$ celdaVacia(t,i,j) $\land$ menorFilaVacia (t,i) $\land$ menorColumnaDeLaFilaVacia(t,i,j)
				
				\pred{menorFilaVacia}{\textbf{t:} \matriz{\ent}, \textbf{i:} \ent}
				{\\
					(\forall f: \ent)(\forall g: \ent) enRango(t,f) $\yLuego$ enRango(t[f],g)\\
					$\implicaLuego$ celdaVacia(t,f,g) $\land$ f \geq i)
				}
		
			\pred{menorColumnaDeLaFilaVacia}{\textbf{t:} \matriz{\ent}, \textbf{i:} \ent, \textbf{j:} \ent}
			{\\
				(\forall g: \ent) enRango(t[i],g)\\
				$\implicaLuego$ celdaVacia(t,i,g) $\land$ g \geq j)
			}
			}
		}
	\end{proc}

5. \begin{proc}{sudoku\_primeraCeldaVaciaColumna}{\textbf{\In t}: \matriz{\ent}, \textbf{\Out result} : \ent }{}{}
		\pre{tableroValido(t)}
    \post
		{
			\IfThenElse{celdasVacias(t) = 0 }
			{-1}
			{
				(\exists i: \ent)(\exists j: \ent) \textbf{result} = j $\land$ enRango(t,i) $\yLuego$ enRango(t[i],j)        $\yLuego$ celdaVacia(t,i,j) $\land$ menorFilaVacia (t,i) $\land$ menorColumnaDeLaFilaVacia(t,i,j)
			}
		}
	\end{proc}

6. \begin{proc}{sudoku\_valorEnCelda}{\textbf{\In t}: \matriz{\ent}, \textbf{\In f}: \ent, \textbf{\In c}: \ent, \textbf{\Out result}: \bool}{}{}
		\pre{tableroValido(t) $\land$ enRango(t,f) $\yLuego$ enRango(t[f],c) $\yLuego$ celdaVacia(t[f][c]) =false }
    \post{\textbf{result} = t[f][c]}
	\end{proc}

7. \begin{proc}{sudoku\_llenarCelda}{\textbf{inout t}: \matriz{\ent} \textbf{\In f}: \ent, \textbf{\In c}: \ent, \textbf{\In value}: \ent}{}{}
    \pre{tableroValido(t) $\land$\\
    0 \leq f, c \leq 8 $\land$\\
    1 \leq value \leq 9 $\land$\\
    t = t_{0} $\land$\\
    t_{0}[f][c] = 0}
    \post{ t = SetAt(t_{0}, c, value)}
	\end{proc}

8. \begin{proc}{sudoku\_vaciarCelda}{\textbf{inout t:} \matriz{\ent}, \textbf{\In f}: \ent, \textbf{\In c}: \ent, \textbf{\Out result:} \bool}{}{}
		\pre{tableroValido(t)}
    \post{}
	\end{proc}

9. \begin{proc}{sudoku\_esTableroParcialmenteResuelto}{\textbf{\In t}: \matriz{\ent}, \textbf{\Out result}: \bool }{}{}
	\pre{\True}
        \post{\textbf{result} = TableroParcialmenteResuelto(t)}
   \end{proc}


10. \begin{proc}{sudoku\_esTableroTotalmenteResuelto}{\textbf{\In t}: \matriz{\ent}, \textbf{\Out result}: \bool}{}{}
		\pre{tableroValido(t)}
    \post{}
	\end{proc}


11. \begin{proc}{sudoku\_esSubTablero}{\textbf{\In t_{0},t_{1}}: \matriz{\ent}, \textbf{\Out result}: \bool}{}{}
		\pre{tableroValido(t_{0}), tableroValido(t_{1})}
    \post
		{
		result = (\forall i: \ent) (\forall j: \ent) 
		}
	\end{proc}

12. \begin{proc}{sudoku\_tieneSolucion}{\textbf{\In t}: \matriz{\ent}, \textbf{\Out tienesolucion}: \bool}{}{}
		\pre{tableroValido(t)}
    \post{}
	\end{proc}


13. \begin{proc}{sudoku\_resolver}{\textbf{inout t}: \matriz{\ent}, \textbf{\Out tienesolucion}: \bool}{}{}
		\pre{\True}
    \post
		{
			\aux{Resolver}{t: \matriz{\ent}}{\matriz{\ent}}
			{\\
				(\exists x: \matriz{\ent})
				\IfThenElse{esSub(t,x) $\yLuego$ tableroTotalmenteResuleto(x)}
				{x}
				{t}
			}
	
		}
	\end{proc}


14. \begin{proc}{sudoku\_copiarTablero}{\textbf{\In src}: \matriz{\ent}, \textbf{\Out target}: \matriz{\ent}}{}{}
       \pre{\True}
       \post
			{
			src = target
			}
    \end{proc}



\section{Predicados y Auxiliares generales}

\pred{Nombre}{t: \matriz{\ent}}{\True}
\pred{PredLargo}{t: \matriz{\ent}}{\\
        (\forall i: \ent)(\forall j: \ent)\True\\
        }
\aux{Aux}{i: \ent}{\bool}{\True}
	
		
			
		\pred{enRango}{t: \TLista{t}, i:\ent}
		{\\
			0 \leq  i < length(t)\\
		}
		
	
		
        
		
\section{Decisiones tomadas}

\end{document}
