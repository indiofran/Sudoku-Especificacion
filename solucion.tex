\documentclass[a4paper]{article} 

\setlength{\parskip}{0.1em}
\input{Algo1Macros}
\usepackage{caratula} % Version modificada para usar las macros de algo1 de ~> https://github.com/bcardiff/dc-tex

\begin{document}

\titulo{TP de Especificaci\'on}
\subtitulo{Sudoku}
\fecha{1 de Abril de 2017}
\materia{Algoritmos y Estructuras de Datos I}
\grupo{Grupo 10}

% Pongan cuantos integrantes quieran
\integrante{Gomez Salaverri, Francisco}{001/01}{francisco@gomezsalaverri.com}
\integrante{Matias Colque, Nadia Noemí}{188/17}{nmatias@dc.uba.ar}
%\integrante{Apellido, Nombre3}{003/01}{email3@dominio.com}
%\integrante{Apellido, Nombre4}{004/01}{email4@dominio.com}

\maketitle

\section{Problemas}
\begin{proc}{sudoku\_esTableroValido}{\textbf{\In t}: \matriz{\ent}, \textbf{\Out result}: \bool}{}{}
    \pre{\True}
    \post{tableroValido(t) = result}
				\pred{tableroValido}{t: \matriz{\ent}}
				{\\
        esFilaValida(t)  $\land$  esColumnaValida(t)\\
        }
				\pred{esFilaValida}{t: \matriz{\ent}}
				{\\
        (\forall i: \ent) (\forall j: \ent)  enRango(t,i)  $\yLuego$\\
				enRango(t[i],j)  $\yLuego$  length(t[i]) = 9  $\implicaLuego$  0 \leq  t[i][j]  \leq 9\\
				}
				\pred{esColumnaValida}{t: \matriz{\ent}}
				{\\
        (\forall i: \ent) (\forall j: \ent) length(t) = 9  $\land$ enRango(t,i) $\yLuego$\\
				enRango(t[i],j)   $\implicaLuego$ 0  \leq  t[i][j] \leq 9\\
				}
\end{proc}


\begin{proc}{sudoku\_esCeldaVacia}{\textbf{\In t}: \matriz{\ent}, \textbf{\In f}: \ent, \textbf{\In c}: \ent, \textbf{ \Out result}: \bool }{}{}
		\pre{tableroValido(t)}
    \post{}
\end{proc}

\begin{proc}{sudoku\_nroDeCeldasVacias}{\textbf{\In t}: \matriz{\ent}, \textbf{\Out result} : \ent }{}{}
		\pre{\True}
    \post{}
\end{proc}

\begin{proc}{sudoku\_primeraCeldaVaciaFila}{\textbf{\In t}: \matriz{\ent}, \textbf{\Out result} : \ent  }{}{}
		\pre{\True}
    \post{}
\end{proc}

\begin{proc}{sudoku\_primeraCeldaVaciaColumna}{\textbf{\In t}: \matriz{\ent}, \textbf{\Out result} : \ent }{}{}
		\pre{\True}
    \post{}
\end{proc}

\begin{proc}{sudoku\_valorEnCelda}{\textbf{\In t}: \matriz{\ent}, \textbf{\In f}: \ent, \textbf{\In c}: \ent, \textbf{\Out result}: \bool}{}{}
		\pre{\True}
    \post{}
\end{proc}

\begin{proc}{sudoku\_llenarCelda}{\textbf{inout t}: \matriz{\ent} \textbf{\In f}: \ent, \textbf{\In c}: \ent, \textbf{\Out result}: \bool}{}{}
		\pre{\True}
    \post{}
\end{proc}

\begin{proc}{sudoku\_vaciarCelda}{\textbf{inout t:} \matriz{\ent}, \textbf{\In f}: \ent, \textbf{\In c}: \ent, \textbf{\Out result:} \bool}{}{}
		\pre{\True}
    \post{}
\end{proc}

\begin{proc}{sudoku\_esTableroParcialmenteResuelto}{\textbf{\In t}: \matriz{\ent}, \textbf{\Out result}: \bool }{}{}
		\pre{\True}
    \post{}
\end{proc}


\begin{proc}{sudoku\_esTableroTotalmenteResuelto}{\textbf{\In t}: \matriz{\ent}, \textbf{\Out result}: \bool}{}{}
		\pre{\True}
    \post{}
\end{proc}


\begin{proc}{sudoku\_esSubTablero}{\textbf{\In t_{0},t_{1}}: \matriz{\ent}, \textbf{\Out result}: \bool}{}{}
		\pre{\True}
    \post{}
\end{proc}

\begin{proc}{sudoku\_tieneSolucion}{\textbf{\In t}: \matriz{\ent}, \textbf{\Out tienesolucion}: \bool}{}{}
		\pre{\True}
    \post{}
\end{proc}


\begin{proc}{sudoku\_resolver}{\textbf{inout t}: \matriz{\ent}, \textbf{\Out tienesolucion}: \bool}{}{}
		\pre{\True}
    \post{}
\end{proc}


\begin{proc}{sudoku\_copiarTablero}{\textbf{\In t}: \matriz{\ent}, \textbf{\Out target}: \matriz{\ent}}{}{}
		\pre{\True}
    \post{}
\end{proc}



\section{Predicados y Auxiliares generales}

\pred{Nombre}{t: \matriz{\ent}}{\True}
\pred{PredLargo}{t: \matriz{\ent}}{\\
        (\forall i: \ent)(\forall j: \ent)\True\\
        }
\aux{Aux}{i: \ent}{\bool}{\True}
	
				
		
\pred{enRango}{t: \TLista{t}, i:\ent}
{\\
	0 \leq  i < length(t)\\
}
		
\aux{Resolver}{t: \matriz{\ent}}{\matriz{\ent}}
{
	\IfThenElse{esSub(t,x) $\yLuego$ tableroParcialmenteResuleto(x)}
	{x}
	{t}
}
		
		
        
		
\section{Decisiones tomadas}

\end{document}
